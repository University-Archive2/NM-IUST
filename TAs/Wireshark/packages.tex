% برای وارد کردن انواع مختلف پسوندهای تصاویر
\usepackage{graphicx}

\usepackage{fancyhdr}
\usepackage{vmargin}
\usepackage{float}

% بسته‌هایی برای کارهای ریاضیاتی
\usepackage{amsmath,amsthm,amssymb}

%برای تنظیم حاشیه
\usepackage{geometry}

% جزو آخرین بسته
\usepackage[hidelinks]{hyperref}

% بسته Xepersian می‌بایست آخرین بسته‌ای باشد که فراخوانی می‌شود.
\usepackage{xepersian}

\usepackage{sectsty}


% تنظیمات geometry
\geometry{left=2cm,right=2.5cm,top=2.5cm,bottom=3cm,a4paper}

%تنظیمات بسته hyperref
\hypersetup{colorlinks=true,linkcolor=blue}

%تنظیم فونت برای xepersian
\settextfont[Scale=1.3]{BNazanin.ttf}

\sectionfont{\huge}
\subsectionfont{\Large}

\setmarginsrb{3 cm}{1.5 cm}{3 cm}{2.5 cm}{1.0 cm}{1.5 cm}{1.0 cm}{1.5 cm}
\setlength{\headheight}{60 pt}

\fancyhead[L]{\Subject}
\fancyhead[C]{\includegraphics[scale = 0.1]{Images/IUSTLogo.png}}
\fancyhead[R]{\Name}


\pagestyle{fancy}

\renewcommand{\labelitemi}{$\bullet$}
\renewcommand{\labelitemii}{$\cdot$}
\renewcommand{\labelitemiii}{$\diamond$}
\renewcommand{\labelitemiv}{$\ast$}
